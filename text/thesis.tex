% structure as seen in https://www.overleaf.com/learn/latex/How_to_Write_a_Thesis_in_LaTeX_(Part_1):_Basic_Structure

\documentclass[12pt]{report}
\usepackage[utf8]{inputenc}
\usepackage{graphicx}
\usepackage{amsmath}
\usepackage{amssymb}
\graphicspath{ {images/} }
\usepackage[ruled,vlined]{algorithm2e}

\usepackage{amsthm}
 
\theoremstyle{definition}
\newtheorem{definition}{Definition}[section]

\title{
{Separation of Foreground and Background Signal in Variational Autoencoders}\\
{\large Albert-Ludwigs-Universität Freiburg}\\[\baselineskip]
{\includegraphics[width=7cm]{logo.png}}
}
\author{Florian Eichin}
\date{02.12.1995}

\begin{document}

\maketitle

\chapter*{}
\begin{center}
To mum and dad.
\end{center}
 
\chapter*{Declaration}
I hereby declare that the thesis submitted is my own unaided work. All direct or indirect sources used are acknowledged as references. This work was not previously presented to another examination board and has not been published. \\ \\ \\

\today
 
\tableofcontents

\chapter{Introduction}

\chapter{Technical Part}

Many ideas of the following chapter are taken from (XX) and the notation mainly follows the same logic. We use lower case bold letters (e.g. $\mathbf{x}$) to denote a vector of random variables and uppercase bold letters (e.g. $\mathbf{X} = \{ x^(1), ..., x^(n)\}$) to denote sets of such random variables, which we also call \emph{datasets}.

\section{Inference Problem Setup}
Before we dive into the technical questions of this thesis, we want to begin with a discussion of the problem we attempt to solve. Let $\mathbf{x}, \mathbf{z}$ be random variables with $\mathbf{x}$ observable and $\mathbf{z}$ hidden. Then we are interested in the \textit{latent variable model} with model parameters $\theta^*$
\begin{equation}
	p_{\mathbf{\theta^*}}(\mathbf{x}, \mathbf{z}) = p_{\mathbf{\theta^*}}(\mathbf{x}| \mathbf{z})p_{\mathbf{\theta^*}}(\mathbf{z})
\end{equation}
We further assume that prior $p_{\mathbf{\theta^*}}(\mathbf{z})$ and $p_{\pmb{\theta^*}}(\mathbf{x}|\mathbf{z})$ are from parametric families of distributions $(p_{\pmb{\theta}}(\mathbf{z}))_{\pmb{\theta} \in \Theta}$ and $(p_{\pmb{\theta}}(\mathbf{x}|\mathbf{z}))_{\pmb{\theta} \in \Theta}$ and that they have probability density functions that are differentiable with respect to $\pmb{\theta}$ and $\mathbf{z}$ almost everywhere.

To make things clearer, we can  look at it in a more practical way: Let $\mathbf{X} = \{ \mathbf{x}^{(i)}\}_{i=1}^N$ be a dataset with $N \in \mathbb{N}$ i.i.d. samples of our random variable $\mathbf{x}$. Note, that $\mathbf{x}$ can be a vector of arbitrary dimension encoding all kinds of data such as images, soundwaves etc. If we model our data with the above latent variable model, we suppose the datapoints to be generated with the involvement of $\mathbf{z}$ in the sense, that first a value $\mathbf{z}^{(i)}$ is generated from prior distribution $p_{\mathbf{\theta^*}}(\mathbf{z})$ and in the second step, $\mathbf{x}^{(i)}$ is generated from $p_{\mathbf{\theta^*}}(\mathbf{x}|\mathbf{z^{(i)}})$. 
Usually, $\mathbf{z}$ is assumed to have a much lower dimension and a much simpler distribution than $\mathbf{x}$. Therefore, the $\mathbf{z}$-space can be viewed as a space of encodings, where only relevant information for decoding datapoints into the high-dimensional $\mathbf{x}$-space is retained. This is why, from a machine learning perspective, we can view the problem of approximating parameters $\pmb{\theta}^*$ as a dimensionality reduction problem.
For a given dataset, there is different approaches for the above scenario. However, we do make additional assumptions, that narrow the list of efficient algorithms significantly [XX]:

\begin{itemize}
	\item[1] \emph{Intractability}: the integral of the marginal likelihood $p_{\mathbf{\theta}}(\mathbf{x}) = \int p_{\mathbf{\theta}}(\mathbf{z}) p_{\mathbf{\theta}}(\mathbf{x}|\mathbf{z}) d \mathbf{z}$, as well as posterior distribution $p_{\mathbf{\theta}}(\mathbf{z}|\mathbf{x}) = p_{\mathbf{\theta}}(\mathbf{x}|\mathbf{z}) p_{\mathbf{\theta}}(\mathbf{z}) / p_{\mathbf{\theta}}(\mathbf{x})$ are intractable.
	\item[2] \emph{Big dataset}: Batch optimization is too expensive and parameter updates on small minibatches preferable. Sampling-based solutions would be too inefficient [XX].
\end{itemize}

\section{Variational Inference}
In many probabilistic models inference is intractable and approximation methods are needed. One way of approximating solutions to inference problems is to describe it as an optimization problem. Algorithms involving this approach are called \emph{variational}.
Given an intractable probability distribution $p_{\pmb{\theta}}$ with parameters from a family of intractable distributions $(p_{\pmb{\theta}})_{\pmb{\theta} \in \pmb{\Theta}}$ and a family of tractable distributions $(q_{\pmb{\phi}})_{\pmb{\phi} \in \pmb{\Phi}}$, the goal of a variational algorithm is to find $\pmb{\phi} \in \pmb{\Phi}$ such that $q_{\pmb{\phi}}$ that is most 'similar' to $p_{\pmb{\theta}}$. Finding such parameters $\pmb{\phi}$ usually involves a complex optimization procedure for an optimization target $\mathcal{L}_{\pmb{\phi}}$. Subsequently, we can use $q_{\pmb{\phi}}$ instead of $p_{\pmb{\theta}}$ in order to find approximate solutions to inference problems efficiently.\\
Of course, this rather informal description on variational techniques leaves us with questions. What is the similarity of two distributions $q_{\pmb{\phi}}$ and $p_{\pmb{\theta}}$? How do we choose an according optimization objective $\mathcal{L}_{\pmb{\phi}}$? What are good ways of formulating tractable family of distributions $(q_{\pmb{\phi}})_{\pmb{\phi} \in \pmb{\Phi}}$? \\
The (partial) answering to these four questions will be the main motivation for the following sections in order to lay the groundwork for the introduction of the Variational Autoencoder (VAE). Inheriting it's name from \emph{Autoencoders}[XX], a VAE is a probabilistic model designed for learning latent representations with the help of Artificial Neuronal Networks (ANNs) and a variational optimization approach to the approximation of the encoder distribution.

\section{Kullback-Leibler divergence}
Beggining with our first question, there is a way of quantifying the 'similarity' of two distributions $p(\mathbf{x})$ and $q(\mathbf{x})$ in information theory known as the \emph{Kullback-Leibler (KL) divergence}. For $p(\mathbf{x}), q(\mathbf{x})$ continuous, the KL divergence is defined as:
\begin{equation}
	KL(q(\mathbf{x})||p(\mathbf{x})) = \int_{-\infty}^{\infty} q(\mathbf{x}) \log \frac{q(\mathbf{x})}{p(\mathbf{x})} d \mathbf{x} = \mathbb{E}_{q(\mathbf{x})}\left[ \log\left(\frac{q(\mathbf{x})}{p(\mathbf{x})} \right) \right]
\end{equation}
In the discrete case, it is analogously 
\begin{equation}
	KL(q(\mathbf{x})||p(\mathbf{x})) = \sum_{\mathbf{x}} q(\mathbf{x}) \log \frac{q(\mathbf{x})}{p(\mathbf{x})} = \mathbb{E}_{q(\mathbf{x})}\left[ \log\left(\frac{q(\mathbf{x})}{p(\mathbf{x})} \right) \right]
\end{equation}
Here, $\log$ is an abbreviation for the logarithm to base $e$, the natural logarithm. Note, that for any $q(\mathbf{x}), p(\mathbf{x})$ (continuous or discrete) we can deduce $KL(q||p) \geq 0$.
Consider $q(\mathbf{x}), p(\mathbf{x})$ to be continuous (the discrete case follows analogously). With $1 - r \leq -\log(r)$ we have:
\begin{equation}
\begin{split}
KL(q(\mathbf{x})||p(\mathbf{x})) 
& = \int_{\mathbf{x}} q(\mathbf{x}) \log \frac{q(\mathbf{x})}{p(\mathbf{x})} d \mathbf{x} \\
& = \int_{\mathbf{x}} q(\mathbf{x}) (- \log \frac{p(\mathbf{x})}{q(\mathbf{x})}) d \mathbf{x} \\
& \geq \int_{\mathbf{x}} q(\mathbf{x}) (1 - \frac{p(\mathbf{x})}{q(\mathbf{x})}) d \mathbf{x} \\
& = \int_{\mathbf{x}} q(\mathbf{x}) d \mathbf{x} - \int_{\mathbf{x}} q(\mathbf{x}) \frac{p(\mathbf{x})}{q(\mathbf{x})} d \mathbf{x} = 0 
\end{split}
\end{equation}

\section{Variational Lower Bound (ELBO)}
As we discussed previously, $p_{\mathbf{\theta}}(\mathbf{x})$ as well as $p_{\mathbf{\theta}}(\mathbf{z}|\mathbf{x})$ are intractable in our problem setup. We have thus no way of retrieving either of the two out of the other. This is where the variational component from two sections before comes into play. In order to approximate $p_{\mathbf{\theta}}(\mathbf{z}|\mathbf{x})$ we introduce a tractable \emph{parametric inference model} $q_{\mathbf{\phi}}(\mathbf{z}|\mathbf{x})$. We will optimize the so called \emph{variational parameters} $\pmb{\phi}$ of this model such that $p_{\mathbf{\theta}}(\mathbf{z}|\mathbf{x}) \approx q_{\mathbf{\phi}}(\mathbf{z}|\mathbf{x})$.
Derived from Bayes' rule, we also have
\begin{equation}
	p_{\mathbf{\theta}}(\mathbf{x}) = \frac{p_{\mathbf{\theta}}(\mathbf{z}|\mathbf{x}) p_{\mathbf{\theta}}(\mathbf{z})}{p_{\mathbf{\theta}}(\mathbf{x}|\mathbf{z})} \approx  \frac{p_{\mathbf{\theta}}(\mathbf{z}|\mathbf{x}) p_{\mathbf{\theta}}(\mathbf{z})}{q_{\mathbf{\phi}}(\mathbf{x}|\mathbf{z})}
\end{equation}
It is clear, that for our model to fit the true distribution of our data well, we are interested in the following two things:
\begin{itemize}
	\item[1.] Maximization of the marginal likelihood $p_{\mathbf{\theta}}(\mathbf{x})$ for our data to improve our generative model.
	\item[2.] Minimization of the KL divergence between $p_{\mathbf{\theta}}(\mathbf{x})$ and $q_{\mathbf{\phi}}(\mathbf{x})$ to improve the approximation of $q_{\mathbf{\phi}}(\mathbf{x})$.
\end{itemize}
Since $\log$ is monotonous, maximizing $p_{\mathbf{\theta}}(\mathbf{x})$ is equivalent to maximizing $\log p_{\mathbf{\theta}}(\mathbf{x})$. For an arbitrary choice of $q_{\mathbf{\phi}}(\mathbf{z}|\mathbf{x})$ we can consider the following derivation:
\begin{equation}
\begin{split}
	\log p_{\mathbf{\theta}}(\mathbf{x}) 
	& = \mathbb{E}_{q_{\mathbf{\phi}}(\mathbf{z}|\mathbf{x})}\left[\log p_{\mathbf{\theta}}(\mathbf{x})\right] \\
	& = \mathbb{E}_{q_{\mathbf{\phi}}(\mathbf{z}|\mathbf{x})}\left[ \log \frac{p_{\mathbf{\theta}}(\mathbf{x}, \mathbf{z})}{p_{\mathbf{\theta}}(\mathbf{z}|\mathbf{x})} \right] \\
	& = \mathbb{E}_{q_{\mathbf{\phi}}(\mathbf{z}|\mathbf{x})}\left[ \log\left(\frac{p_{\mathbf{\theta}}(\mathbf{x}, \mathbf{z})}{q_{\mathbf{\phi}}(\mathbf{z}|\mathbf{x})}\frac{q_{\mathbf{\phi}}(\mathbf{z}|\mathbf{x})}{p_{\mathbf{\theta}}(\mathbf{z}|\mathbf{x})} \right) \right] \\
	& = \mathbb{E}_{q_{\mathbf{\phi}}(\mathbf{z}|\mathbf{x})}\left[ \log\left(\frac{p_{\mathbf{\theta}}(\mathbf{x}, \mathbf{z})}{q_{\mathbf{\phi}}(\mathbf{z}|\mathbf{x})}\right) \right] + \mathbb{E}_{q_{\mathbf{\phi}}(\mathbf{z}|\mathbf{x})}\left[ \log\left(\frac{q_{\mathbf{\phi}}(\mathbf{z}|\mathbf{x})}{p_{\mathbf{\theta}}(\mathbf{z}|\mathbf{x})} \right) \right] \\
	& = \mathbb{E}_{q_{\mathbf{\phi}}(\mathbf{z}|\mathbf{x})}\left[ \log\left(\frac{p_{\mathbf{\theta}}(\mathbf{x}, \mathbf{z})}{q_{\mathbf{\phi}}(\mathbf{z}|\mathbf{x})}\right) \right] + KL(q_{\mathbf{\phi}}(\mathbf{z}|\mathbf{x}) || p_{\mathbf{\theta}}(\mathbf{z}| \mathbf{x})) 
\end{split}
\end{equation}
Where the right term in the last row is the KL divergence of $q_{\mathbf{\phi}}(\mathbf{z}|\mathbf{x})$ and $p_{\mathbf{\theta}}(\mathbf{x}, \mathbf{z})$. If we rearrange the equation, we have the following:
\begin{equation}
	\log p_{\mathbf{\theta}}(\mathbf{x}) - KL(q_{\mathbf{\phi}}(\mathbf{z}|\mathbf{x}) || p_{\mathbf{\theta}}(\mathbf{z}| \mathbf{x})) = \mathbb{E}_{q_{\mathbf{\phi}}(\mathbf{z}|\mathbf{x})}\left[ \log\left(\frac{p_{\mathbf{\theta}}(\mathbf{x}, \mathbf{z})}{q_{\mathbf{\phi}}(\mathbf{z}|\mathbf{x})}\right) \right]
\end{equation}
And since $KL(q_{\mathbf{\phi}}(\mathbf{z}|\mathbf{x}) || p_{\mathbf{\theta}}(\mathbf{x}, \mathbf{z})) \geq 0$, the right hand side is a lower bound for $\log p_{\mathbf{\theta}}(\mathbf{x})$. It is also referred to as \emph{variational lower bound} or \emph{evidence lower bound} (ELBO):
\begin{equation}
\begin{split}
	\mathcal{L}_{\mathbf{\theta}, \mathbf{\phi}}(\mathbf{x}) 
	& = \mathbb{E}_{q_{\mathbf{\phi}}(\mathbf{z}|\mathbf{x})}\left[ \log\left(\frac{p_{\mathbf{\theta}}(\mathbf{x}, \mathbf{z})}{q_{\mathbf{\phi}}(\mathbf{z}|\mathbf{x})}\right) \right] \\
	& = \mathbb{E}_{q_{\mathbf{\phi}}(\mathbf{z}|\mathbf{x})}\left[ \log p_{\mathbf{\theta}}(\mathbf{x}, \mathbf{z}) - \log q_{\mathbf{\phi}}(\mathbf{z}|\mathbf{x}) \right] \\\\
\end{split}
\end{equation}
We will stick to this form of the ELBO for the rest of this work, but we want to note that in can be rewritten in the following way:
\begin{equation}
\begin{split}
& \mathbb{E}_{q_{\mathbf{\phi}}(\mathbf{z}|\mathbf{x})}\left[ \log\left(\frac{p_{\mathbf{\theta}}(\mathbf{x}, \mathbf{z})}{q_{\mathbf{\phi}}(\mathbf{z}|\mathbf{x})}\right) \right] \\
& = \mathbb{E}_{q_{\mathbf{\phi}}(\mathbf{z}|\mathbf{x})}\left[ \log\left(\frac{p_{\mathbf{\theta}}(\mathbf{x}|\mathbf{z})p_{\mathbf{\theta}}(\mathbf{z})}{q_{\mathbf{\phi}}(\mathbf{z}|\mathbf{x})}\right) \right] \\
	& = \mathbb{E}_{q_{\mathbf{\phi}}(\mathbf{z}|\mathbf{x})}\left[ \log p_{\mathbf{\theta}}(\mathbf{x}| \mathbf{z}) - \log \frac{q_{\mathbf{\phi}}(\mathbf{z}|\mathbf{x})}{p_{\mathbf{\theta}}(\mathbf{z})} \right]	\\
	& = \mathbb{E}_{q_{\mathbf{\phi}}(\mathbf{z}|\mathbf{x})}\left[ \log p_{\mathbf{\theta}}(\mathbf{x}| \mathbf{z})\right] - \mathbb{E}_{q_{\mathbf{\phi}}(\mathbf{z}|\mathbf{x})}\left[\log \frac{q_{\mathbf{\phi}}(\mathbf{z}|\mathbf{x})}{p_{\mathbf{\theta}}(\mathbf{z})} \right]	\\
	& = \mathbb{E}_{q_{\mathbf{\phi}}(\mathbf{z}|\mathbf{x})}\left[ \log p_{\mathbf{\theta}}(\mathbf{x}| \mathbf{z})\right] + KL(q_{\mathbf{\phi}}(\mathbf{z}|\mathbf{x})||p_{\mathbf{\theta}}(\mathbf{z}))\\	
\end{split}
\end{equation}
With the above derivation (XX) in mind, we can identify another interpretation of $KL(q_{\mathbf{\phi}}(\mathbf{z}|\mathbf{x}) || p_{\mathbf{\theta}}(\mathbf{z}|\mathbf{x}))$ besides being the KL divergence of approximate posterior $q_{\mathbf{\phi}}(\mathbf{z}|\mathbf{x})$ and true posterior $p_{\mathbf{\theta}}(\mathbf{z}| \mathbf{x}))$: It is also the gap between ELBO $\mathcal{L}_{\mathbf{\theta}, \mathbf{\phi}}(\mathbf{x})$ and $\log p_{\mathbf{\theta}}(\mathbf{x})$. If $q_{\mathbf{\phi}}(\mathbf{z}|\mathbf{x})$ approximates the true $p_{\mathbf{\theta}}(\mathbf{z}|\mathbf{x})$ 'better', the gap gets smaller.

Let $\mathbf{X}$ be the dataset of i.i.d. samples from before and $N_{\mathbf{X}} = |\mathbf{X}|$. If we want to fit our model on $\mathbf{X}$, the ELBO yields us an optimization objective we were asking for, namely the average of ELBOs of single datapoints $\mathbf{x} \in \mathbf{X}$:
\begin{equation}
	\mathcal{L}_{\mathbf{\theta}, \mathbf{\phi}}(\mathbf{X}) = \sum_{\mathbf{x} \in \mathbf{X}} \frac{\mathcal{L}_{\mathbf{\theta}, \mathbf{\phi}}(\mathbf{x})}{N_{\mathbf{X}}}
\end{equation}
If we maximize $\mathcal{L}_{\mathbf{\theta}, \mathbf{\phi}}(\mathbf{x})$ with respect to parameters $\pmb{\theta}$ and $\pmb{\phi}$ for our data, we will approximately maximize $p_{\mathbf{\theta}}(\mathbf{x})$ and minimize $KL(q_{\mathbf{\phi}}(\mathbf{z}|\mathbf{x}) || p_{\mathbf{\theta}}(\mathbf{z}| \mathbf{x}))$ just like the goals we formulated in the beginning of this section.

\section{Auto-encoding Variational Bayes}
\subsection{Batch Gradient Descent}
%TODO
With the means of the ELBO, we now have an objective to optimize the model parameters $\pmb{\theta}$ and $\pmb{\phi}$ for. A naive solution, also known as \emph{Batch Gradient Descent}, is to initialize the parameters randomly, to estimate the gradients $\nabla_{\pmb{\theta}}\mathcal{L}_{\pmb{\theta}, \pmb{\phi}}(\mathbf{X})$ and $\nabla_{\pmb{\pmb{\phi}}}\mathcal{L}_{\pmb{\theta}, \pmb{\phi}}(\mathbf{X})$ and adjust $\pmb{\theta}$ and $\pmb{\phi}$ into their respective directions until convergence. With each step of adjusting the parameters for the gradients, also called an \emph{epoch}, we expect $\mathcal{L}_{\pmb{\theta}, \pmb{\phi}}(\mathbf{X})$ to improve until we have reached a local maximum and the algorithm converges. It is up to implementation how to detect convergence of the algorithm. Typically, one can define criteria such as a threshold for change of loss after a certain number of epochs epoch. If the change of loss is lower than this set threshhold, we abort the procedure. Of course, there is other, more complex criteria. Sometimes the user decides theirselves, when to stop the algorithm due to the tradeoff between runtime and optimality.
\begin{algorithm}[H]
\SetAlgoLined
\While{not converged}{
Estimate gradients $\nabla_{\pmb{\theta}}\mathcal{L}_{\pmb{\theta}, \pmb{\phi}}(\mathbf{X})$, $\nabla_{\pmb{\pmb{\phi}}}\mathcal{L}_{\pmb{\theta}, \pmb{\phi}}(\mathbf{X})$\\
Update parameters $\pmb{\theta} \rightarrow \pmb{\theta} + \eta \nabla_{\pmb{\theta}}\mathcal{L}_{\pmb{\theta}, \pmb{\phi}}(\mathbf{X})$, $\pmb{\phi} \rightarrow \pmb{\phi} + \eta \nabla_{\pmb{\phi}}\mathcal{L}_{\pmb{\theta}, \pmb{\phi}}(\mathbf{X})$
} 
\caption{Batch Gradient Descent}
\end{algorithm} 
Note, that $\eta$ is a hyperparameter, that determines the extent to which the gradients update the parameters for each epoch. It is therefore also called the \emph{learning rate} and is an important parameter to choose as it highly affects the convergence of the algorithm: If we choose $\eta$ too small, we will only move slowly in the 'preferred' direction of local maxima. However, if we choose $\eta$ too big, our updating-steps get too large and the algorithm cannot converge in the worst case. 


\subsection{Estimation of the gradients and ELBO}
Because in general analytical computations of the gradients of the ELBO are intractable, we have to estimate them. For $\nabla_{\pmb{\theta}, \pmb{\phi}}\mathcal{L}_{\pmb{\theta}, \pmb{\phi}}(\mathbf{X})$, it is sufficient to estimate $\nabla_{\pmb{\theta}, \pmb{\phi}}\mathcal{L}_{\pmb{\theta}, \pmb{\phi}}(\mathbf{x})$ for each $\mathbf{x} \in \mathbf{X}$ because since (XX) we have
\begin{equation}
	\nabla_{\pmb{\theta}, \pmb{\phi}}\mathcal{L}_{\pmb{\theta}, \pmb{\phi}}(\mathbf{X}) = \nabla_{\pmb{\theta}, \pmb{\phi}}  \sum_{\mathbf{x} \in \mathbf{X}} \frac{\mathcal{L}_{\mathbf{\theta}, \mathbf{\phi}}(\mathbf{x})}{N_{\mathbf{X}}} = \frac{1}{N_{\mathbf{X}}} \sum_{\mathbf{x} \in \mathbf{X}} \nabla_{\pmb{\theta}, \pmb{\phi}} \mathcal{L}_{\mathbf{\theta}, \mathbf{\phi}}(\mathbf{x})
\end{equation}
For gradients of the ELBO with respect to generative model parameters $\pmb{\theta}$, we can obtain an unbiased Monte Carlo estimator (unbiased estimation will from now on be illustrated by '$\simeq$'):
\begin{equation}
\begin{split}
\nabla_{\pmb{\theta}}\mathcal{L}_{\pmb{\theta}, \pmb{\phi}}(\mathbf{x}) 
& \stackrel{(XX)}{=} \nabla_{\pmb{\theta}} \mathbb{E}_{q_{\mathbf{\phi}}(\mathbf{z}|\mathbf{x})}\left[ \log p_{\mathbf{\theta}}(\mathbf{x}, \mathbf{z}) - \log q_{\mathbf{\phi}}(\mathbf{z}|\mathbf{x}) \right]	\\
& = \mathbb{E}_{q_{\mathbf{\phi}}(\mathbf{z}|\mathbf{x})}\left[ \nabla_{\pmb{\theta}} (\log p_{\mathbf{\theta}}(\mathbf{x}, \mathbf{z}) - \log q_{\mathbf{\phi}}(\mathbf{z}|\mathbf{x})) \right] \\
& \simeq \nabla_{\pmb{\theta}} (\log p_{\mathbf{\theta}}(\mathbf{x}, \mathbf{z}) - \log q_{\mathbf{\phi}}(\mathbf{z}|\mathbf{x})) \\
& = \nabla_{\pmb{\theta}} \log p_{\mathbf{\theta}}(\mathbf{x}, \mathbf{z})\\
\end{split}
\end{equation}
With the $\mathbf{z}$ in the last two lines a random sample from $q_{\mathbf{\phi}}(\mathbf{z}|\mathbf{x})$. For unbiased gradients with respect to $\pmb{\phi}$, things are a bit more difficult. This is due to the fact, that in general, we have:
\begin{equation}
\begin{split}
\nabla_{\pmb{\phi}}\mathcal{L}_{\pmb{\theta}, \pmb{\phi}}(\mathbf{x}) 
& = \nabla_{\pmb{\phi}} \mathbb{E}_{q_{\mathbf{\phi}}(\mathbf{z}|\mathbf{x})}\left[ \log p_{\mathbf{\theta}}(\mathbf{x}, \mathbf{z}) - \log q_{\mathbf{\phi}}(\mathbf{z}|\mathbf{x}) \right]	\\
& \neq \mathbb{E}_{q_{\mathbf{\phi}}(\mathbf{z}|\mathbf{x})}\left[ \nabla_{\pmb{\phi}} (\log p_{\mathbf{\theta}}(\mathbf{x}, \mathbf{z}) - \log q_{\pmb{\phi}}(\mathbf{z}|\mathbf{x})) \right] \\
\end{split}
\end{equation}
%TODO : check argument for continuous and diff
Moreover, we will thus assume $\mathbf{z}$ to be continuous and $q_{\mathbf{\phi}}(\mathbf{z}|\mathbf{x})$ and $p_{\mathbf{\theta}}(\mathbf{x}|\mathbf{z})$ differentiable. For the case of the Variational Autoencoder (and other instances of our problem), we are not constrained by this. However, we can now reparameterize $\mathbf{z} \sim q_{\mathbf{\phi}}(\mathbf{z}|\mathbf{x})$ to circumvent the problem of obtaining $\nabla_{\pmb{\phi}}\mathcal{L}_{\pmb{\theta}, \pmb{\phi}}(\mathbf{x})$. We can choose $f$ as some invertible, differentiable transformation and introduce another random variable $\pmb{\epsilon} \sim p(\pmb{\epsilon})$ independent of $\pmb{\phi}$, $\pmb{\theta}$ and $\mathbf{x}$ such that
\begin{equation}
\mathbf{z} = f(\pmb{\epsilon}, \pmb{\phi}, \mathbf{x})
\end{equation}
This is also referred to as \emph{reparameterization trick}. Under reparameterization, the ELBO can be rewritten:
\begin{equation}
\begin{split}
\mathcal{L}_{\pmb{\theta}, \pmb{\phi}}(\mathbf{x}) 
& = \mathbb{E}_{q_{\mathbf{\phi}}(\mathbf{z}|\mathbf{x})}\left[ \log p_{\mathbf{\theta}}(\mathbf{x}, \mathbf{z}) - \log q_{\mathbf{\phi}}(\mathbf{z}|\mathbf{x})\right]\\
& = \mathbb{E}_{p(\pmb{\epsilon})}\left[ \log p_{\mathbf{\theta}}(\mathbf{x}, f(\pmb{\epsilon}, \pmb{\phi}, \mathbf{x})) - \log q_{\mathbf{\phi}}(f(\pmb{\epsilon}, \pmb{\phi}, \mathbf{x})|\mathbf{x}) \right]	\\
& \simeq \log p_{\mathbf{\theta}}(\mathbf{x}, f(\pmb{\epsilon}, \pmb{\phi}, \mathbf{x})) - \log q_{\mathbf{\phi}}(f(\pmb{\epsilon}, \pmb{\phi}, \mathbf{x})|\mathbf{x})\\
& =: \tilde{\mathcal{L}}_{\pmb{\theta}, \pmb{\phi}}(\mathbf{x}, \pmb{\epsilon})\\
\end{split}
\end{equation}
Where we used a single sample $\pmb{\epsilon}\sim p(\pmb{\epsilon})$ to derive the Monte Carlo estimator $\tilde{\mathcal{L}}_{\pmb{\theta}, \pmb{\phi}}(\mathbf{x}, \pmb{\epsilon})$. We can rewrite our gradients and  gain an estimator in a similar fashion:
\begin{equation}
\begin{split}
\nabla_{\pmb{\theta}, \pmb{\phi}}\mathcal{L}_{\pmb{\theta}, \pmb{\phi}}(\mathbf{x}) 
& = \nabla_{\pmb{\theta}, \pmb{\phi}} \mathbb{E}_{p(\pmb{\epsilon})}\left[ \log p_{\mathbf{\theta}}(\mathbf{x}, f(\pmb{\epsilon}, \pmb{\phi}, \mathbf{x})) - \log q_{\mathbf{\phi}}(f(\pmb{\epsilon}, \pmb{\phi}, \mathbf{x})|\mathbf{x}) \right]	\\
& = \mathbb{E}_{p(\pmb{\epsilon})}\left[ \nabla_{\pmb{\theta}, \pmb{\phi}}(\log p_{\mathbf{\theta}}(\mathbf{x}, f(\pmb{\epsilon}, \pmb{\phi}, \mathbf{x})) - \log q_{\mathbf{\phi}}(f(\pmb{\epsilon}, \pmb{\phi}, \mathbf{x})|\mathbf{x})) \right]	\\
& \simeq \nabla_{\pmb{\theta}, \pmb{\phi}}(\log p_{\mathbf{\theta}}(\mathbf{x}, f(\pmb{\epsilon}, \pmb{\phi}, \mathbf{x})) - \log q_{\mathbf{\phi}}(f(\pmb{\epsilon}, \pmb{\phi}, \mathbf{x})|\mathbf{x})) \\
& = \nabla_{\pmb{\theta}, \pmb{\phi}}\tilde{\mathcal{L}}_{\pmb{\theta}, \pmb{\phi}}(\mathbf{x}, \pmb{\epsilon})
\end{split}
\end{equation}
Again with the last two rows a Monte Carlo estimator for a single sample $\pmb{\epsilon} \sim p(\pmb{\epsilon})$. 
Both estimators are unbiased because:
%TODO : add sums for expectations 
\begin{equation}
\begin{split}
\mathbb{E}_{p(\pmb{\epsilon})}\left[ \nabla_{\pmb{\theta}, \pmb{\phi}}\tilde{\mathcal{L}}_{\pmb{\theta}, \pmb{\phi}}(\mathbf{x}, \pmb{\epsilon}) \right] 
& = \mathbb{E}_{p(\pmb{\epsilon})}\left[ \nabla_{\pmb{\phi}, \pmb{\theta}} \log p_{\mathbf{\theta}}(\mathbf{x}, f(\pmb{\epsilon}, \pmb{\phi}, \mathbf{x})) - \log q_{\mathbf{\phi}}(f(\pmb{\epsilon}, \pmb{\phi}, \mathbf{x})|\mathbf{x}) \right] \\
& = \nabla_{\pmb{\phi}, \pmb{\theta}} \mathbb{E}_{p(\pmb{\epsilon})}\left[  \log p_{\mathbf{\theta}}(\mathbf{x}, f(\pmb{\epsilon}, \pmb{\phi}, \mathbf{x})) - \log q_{\mathbf{\phi}}(f(\pmb{\epsilon}, \pmb{\phi}, \mathbf{x})|\mathbf{x}) \right] \\
& = \nabla_{\pmb{\phi}, \pmb{\theta}} \mathbb{E}_{q_{\mathbf{\phi}}(\mathbf{z}|\mathbf{x})}\left[  \log p_{\mathbf{\theta}}(\mathbf{x}, \mathbf{z}) - \log q_{\mathbf{\phi}}(\mathbf{z}|\mathbf{x}) \right] \\
& = \nabla_{\pmb{\theta}, \pmb{\phi}}\mathcal{L}_{\pmb{\theta}, \pmb{\phi}}(\mathbf{x}) 
\end{split}
\end{equation}
Which works with an analog argument for $\tilde{\mathcal{L}}_{\pmb{\theta}, \pmb{\phi}}(\mathbf{x}, \pmb{\epsilon})$. In order to compute this estimation of the ELBO and its gradients, we will need to compute $\log q_{\mathbf{\phi}}(\mathbf{z}|\mathbf{x})=\log q_{\mathbf{\phi}}(f(\pmb{\epsilon}, \pmb{\phi}, \mathbf{x})|\mathbf{x})$. 
We will use a result from Statistics, called the \emph{Change-of-Variables Technique} [XX], which yields us 
\begin{equation}
q_{\mathbf{\phi}}(f(\pmb{\epsilon}, \pmb{\phi}, \mathbf{x})|\mathbf{x}) \left|\det \left(\frac{\partial}{\partial \pmb{\epsilon}}f(\pmb{\epsilon}, \pmb{\phi}, \mathbf{x})\right)\right| = p(\pmb{\epsilon})
\end{equation}
Where $\left|\det \left(\frac{\partial}{\partial \pmb{\epsilon}}f(\pmb{\epsilon}, \pmb{\phi}, \mathbf{x})\right)\right|$ is the absolute value of the determinant of the Jacobian. Putting it together, we thus have: 
\begin{equation}
\begin{split}
\log q_{\mathbf{\phi}}(f(\pmb{\epsilon}, \pmb{\phi}, \mathbf{x})|\mathbf{x}) & = \log p(\pmb{\epsilon}) - \log \left|\det \left(\frac{\partial}{\partial \pmb{\epsilon}}f(\pmb{\epsilon}, \pmb{\phi}, \mathbf{x})\right)\right| \\
\end{split}
\end{equation}
For many choices of reparameterizations, this is simple to compute, as we will see later.


\subsection{Stochastic Gradient Descent} 
Vanilla Gradient Descent has several shortcomings, that we still have to address. For one, it is usually very expensive to estimate $\nabla_{\pmb{\theta}, \pmb{\phi}}\mathcal{L}_{\pmb{\theta}, \pmb{\phi}}(\mathbf{X})$ on big datasets $\mathbf{X}$, violating our goal for efficiency with large data. Also, the algorithm usually only finds the closest local maximum and has no means to escape before convergence. There is several solutions to those problems. One that tackles both is called \emph{Minibatch Stochastic Gradient Descent (SGD)}. 
The idea behind Minibatch SGD is again intuitive: Instead of estimating the gradients on the whole dataset $\mathbf{X}$, we randomly draw a subset $\mathbf{M} \subset \mathbf{X}$ of size $N_{\mathbf{M}}$. We call $\mathbf{M}$ a \emph{minibatch}. We have:
\begin{equation}
\nabla_{\pmb{\theta}, \pmb{\phi}}\mathcal{L}_{\pmb{\theta}, \pmb{\phi}}(\mathbf{X}) 
= \frac{1}{N_{\mathbf{X}}} \sum_{\mathbf{x} \in \mathbf{X}} \nabla_{\pmb{\theta}, \pmb{\phi}} \mathcal{L}_{\mathbf{\theta}, \mathbf{\phi}}(\mathbf{x}) 
\simeq \frac{1}{N_{\mathbf{M}}} \sum_{\mathbf{x} \in \mathbf{M}} \nabla_{\pmb{\theta}, \pmb{\phi}} \mathcal{L}_{\mathbf{\theta}, \mathbf{\phi}}(\mathbf{x})
= \nabla_{\pmb{\theta}, \pmb{\phi}}\mathcal{L}_{\pmb{\theta}, \pmb{\phi}}(\mathbf{M}) 
\end{equation}
Assume $\mathbf{M}'=\{(\mathbf{x}, \pmb{\epsilon}) |  \mathbf{x} \in \mathbf{M}, \pmb{\epsilon} \sim p(\pmb{\epsilon})\}$ to be a set of tuples of each datapoint in $\mathbf{M}$ and an according sample of $\pmb{\epsilon}$. We can get the following unbiased gradient estimators from combining (XX) and (XX):
\begin{equation}
\begin{split}
& \nabla_{\pmb{\theta}, \pmb{\phi}}\mathcal{L}_{\pmb{\theta}, \pmb{\phi}}(\mathbf{M}) \\
&	\simeq \frac{1}{N_{\mathbf{M}}} \sum_{(\mathbf{x}^{(i)}, \pmb{\epsilon}^{(i)}) \in \mathbf{M}'} \nabla_{\pmb{\theta}, \pmb{\phi}}(\log p_{\mathbf{\theta}}(\mathbf{x}^{(i)}, f(\pmb{\epsilon}^{(i)}, \pmb{\phi}, \mathbf{x})) - \log q_{\mathbf{\phi}}(f(\pmb{\epsilon}^{(i)}, \pmb{\phi}, \mathbf{x}^{(i)})|\mathbf{x}^{(i)}))
\end{split}
\end{equation}
Usually the minibatch size $N_{\mathbf{M}}$ is set to be much lower than the number of datapoints in our dataset. Depending on application domain and optimization target, different sizes are optimal. For $N_\mathbf{M}=1$ we also have 'normal' \emph{Stochastic Gradient Descent} as an instance of Minibatch SGD. It is easy to see, why Minibatch SGD is a more efficient way of optimizing our model parameters: The cost of estimating our gradients on smaller minibatches is of course much cheaper than on the whole dataset. In the same runtime, Minibatch SGD updates parameters much more often than vanilla Gradient Descent, usually leading to much faster convergence. Besides bare runtime, it also saves memory as we only have to load a small fraction of the dataset in order for the algorithm to work. Also, because we update our parameters on random subsets, the optimization process becomes more 'noisy', allowing the algorithm to escape local maxima by taking globally non-optimal steps. However, this does not solve all the problems with local maxima. Theres more techniques to tackle this problem, but an in-depth discussion would by far escape the scope of this thesis. The interested reader is advised to continue their research in [XX].

\subsection{Stochastic Optimization of the ELBO}
Putting things together, we can now introduce the \emph{Auto-Encoding Variational Bayes} procedure, that utilizes Minibatch SGD and the gradient estimators we have derived.

\begin{algorithm}[H]
\SetAlgoLined
Initialize parameters $\pmb{\theta}$, $\pmb{\phi}$ randomly\\
\While{not converged}{
Randomly draw a minibatch $\mathbf{M} \subset \mathbf{X}$\\
Sample $\pmb{\epsilon}_1, ..., \pmb{\epsilon}_{N_\mathbf{M}} \sim p(\pmb{\epsilon})$\\
Compute $\tilde{\mathcal{L}}_{\pmb{\theta}, \pmb{\phi}}(\mathbf{M})$ and gradient $\nabla_{\pmb{\theta}, \pmb{\phi}}\tilde{\mathcal{L}}_{\pmb{\theta}, \pmb{\phi}}(\mathbf{M})$\\
Update parameters $\pmb{\theta}$, $\pmb{\phi}$
}\caption{Auto-Encoding Variational Bayes (AEVB)}
\end{algorithm}
Note, how not only the minibatch drawing of SGD, but also the sampling of $\epsilon \sim p(\epsilon)$ introduces noise to the procedure. This makes it even more robust to getting stuck in local maxima, for the reasosns discussed above. Also, we have replaced the parameter updating in order to account for more complex strategies of different instances of this algorithm. While simply updating the parameters in the direction of the gradients with a set learning rate $\eta$ is still a viable approach, there is other variations that can lead to better results for different application domains. A more detailed discussion can be found in [XX].


\section{Artificial Neural Networks}
\emph{Artificial Neural Networks} (ANN) have gained a lot of attention due to their stellar performance in many different domains of machine learning lately. An ANN is a function composed by \emph{layers} $\alpha^{(i)}$: 
\begin{equation}
NeuralNet = \alpha^{(L)} \circ \alpha^{(L-1)} \circ ... \circ \alpha^{(1)}
\end{equation}
With $L$ the number of layers. For each $i \in \{2, ..., L\} $ we define \emph{weight matrix} $W^{(i)} \in Mat(M^{(i)} \times N^{(i)}, \mathbb{R})$, \emph{bias vector} $b^{(i)} \in \mathbb{R}^{M^{(i)}}$ and \emph{activation function} $g^{(i)}: \mathbb{R} \rightarrow \mathbb{R}$. Layer $\alpha^{(i)}$ maps a vector of \emph{input dimension} $N^{(i)}$ to a vector of \emph{output dimension} $N^{(i+1)}$ as follows:
\begin{equation}
\begin{split}
\alpha^{(i)}: \mathbb{R}^{N^{(i)}} & \rightarrow \mathbb{R}^{N^{(i+1)}} \\
x & \mapsto g^{(i)}(W^{(i)}x+b^{(i)})\\
\end{split}
\end{equation}
These are also called the \emph{hidden layers} of our network and $\alpha^{(L)}$ is the \emph{output layer}. By convention, we define \emph{input layer} $\alpha^{(1)}$ as the identity of the input vector of the network and $N^{(1)}$ as its dimension.
Usually the activation functions $g^{(i)}$, which are applied element-wise, are chosen to be non-linear, or else $\alpha$ would just be a linear function and the ANN would deflate into a single linear transformation as well. 
In order to enable an ANN to 'learn' $\pmb{\phi} = \{W^{(2)},  b^{(2)}, W^{(3)},  b^{(3)}, ..., W^{(L)},  b^{(L)}\}$ are interpreted as parameters, that we can update with respect to their gradients. Therefore, $g^{(i)}$ has to be differentiable (almost) everywhere (we have to define assumed values for the derivation at points where $g^{(i)}$ is non-differntiable). Common activation functions include many others:
\begin{itemize}
\item sigmoid function $Sig(x) = \frac{1}{1 - \exp(-kx)}$ with $k \in \mathbb{R}$
\item rectifier function $ReLu(x) = \max(0, x)$ (with $ReLu'(0) := 0$)
\item tangens hyperbolicus $tanh$
\end{itemize} 
And while each has it's own perks, some are better suited for different tasks. For example, in practice, a big advantage of $ReLu$ is the faster convergence of optimization tasks. On the other hand, $tanh$ and historically important $Sig$ sometimes offer better interpretability of the results as they have values in $(0, 1)$. \\
The gradients with respect to the parameters $\pmb{\phi}$ of a NeuralNetwork can be calculated using a procedure called \emph{Backpropagation}, which exploits the chain rule of differentiation. Another approach that is currently evolving is \emph{differentiable programming}, where the key idea is to compute derivatives for defined functions automatically at compiling time. In the case of neural networks, there is only matrix multiplications, vector additions and the usually simple activation functions which makes them a promising application for differential programming.

\section{The Variational Auto-encoder (VAE)}
\subsection{Choice of model}

The definition of the AEVB algorithm left some options regarding the exact nature of the final model. Namely, we still have to choose:
\begin{itemize}
\item[1)] the parameterization of decoder $p_{\pmb{\theta}}(\mathbf{x}|\mathbf{z})$
\item[2)] the distribution of prior $p_{\pmb{\theta}}(\mathbf{z})$ and its reparameterization $\mathbf{z} = f(\pmb{\epsilon}, \pmb{\phi}, \mathbf{x})$
\item[3)] the parameterization for the variational encoder $q_{\pmb{\phi}}(\mathbf{z}|\mathbf{x})$
\end{itemize}
We want 1) to be complex enough to model the data sufficiently well. 3) brings us back to our last question from the beginning of this chapter, where we asked for a good way to formulate a the variational class of tractable distributions. 
 

\subsection{Neural Networks for parameterizing distributions}
For the VAE, we let $p_{\pmb{\theta}}(\mathbf{x}|\mathbf{z})$ be a multivariate normal Distribution (for continuous data) or Bernoulli (binary data). We can model the parameters $\pmb{\mu}, \pmb{\Sigma}$ or $\mathbf{p}$ for both cases as the outputs of an ANN with parameters $\pmb{\theta}$:
\begin{equation}
\begin{split}
p_{\pmb{\theta}}(\mathbf{x}|\mathbf{z}) & = \mathcal{N}(\pmb{\mu}, \pmb{\Sigma})(\mathbf{z}) \ \mathrm{or} \ \mathcal{B}(\mathbf{p})(\mathbf{z}) \\
(\pmb{\mu}, \pmb{\Sigma}) \ \mathrm{or} \ \mathbf{p} & = DecoderANN_{\pmb{\theta}}(\mathbf{x}) \\
\end{split}
\end{equation}
Furthermore, we let $p_{\pmb{\theta}}(\mathbf{z})=\mathcal{N}(0, E)(\mathbf{x})$ with $E=\mathrm{diag}(1)$ be the centered isotropic multivariate normal Distribution. In this case,  $p_{\pmb{\theta}}(\mathbf{z}|\mathbf{x})$ is intractable and we choose $q_{\pmb{\phi}}(\mathbf{z}|\mathbf{x})=\mathcal{N}(\pmb{\mu}, \pmb{\sigma}^2 E)(\mathbf{x})$ assuming the true posterior to approximately be a multivariate normal distribution with diagonal covariance. Again, we can use an ANN to parameterize mean $\pmb{\mu}$ and log-deviation $\log \pmb{\sigma}$ of the variational posterior in the following sense: 
\begin{equation}
\begin{split}
q_{\pmb{\phi}}(\mathbf{z}|\mathbf{x}) & = \mathcal{N}(\pmb{\mu}, \mathrm{diag}(\pmb{\sigma})^2)(\mathbf{z}) \\
(\pmb{\mu}, \log \pmb{\sigma}) & = NeuralNet_{\pmb{\phi}}(\mathbf{x}) \\
\end{split}
\end{equation}
% Note, that with these assumptions, we have:
%\begin{equation}
%\begin{split}
%q_{\pmb{\phi}}(\mathbf{z}|\mathbf{x}) & = \prod_i q_{\pmb{\phi}}(\mathbf{z}_i|\mathbf{x}) = \prod_i \mathcal{N}(\pmb{\mu}_i, \pmb{\sigma}_i^2)(\mathbf{z}_i)\\
%\end{split}
%\end{equation}
Reparameterization in this case is simple:
\begin{equation}
\begin{split}
\pmb{\epsilon} & \sim \mathcal{N}(0, E) \\
(\pmb{\mu}, \log \pmb{\sigma}) & = NeuralNet_{\pmb{\phi}}(\mathbf{x}) \\
\mathbf{z} & = f(\pmb{\epsilon}, \pmb{\phi}, \pmb{x}) = \pmb{\mu} + \pmb{\sigma} \cdot \pmb{\epsilon}\\
\end{split}
\end{equation}
with $\cdot$ the element-wise multiplication. And the Jacobian from $\pmb{\epsilon}$ to $\mathbf{z} = f(\pmb{\epsilon}, \pmb{\phi}, \pmb{x})$ is:
\begin{equation}
\frac{\partial}{\partial \pmb{\epsilon}}f(\pmb{\epsilon}, \pmb{\phi}, \mathbf{x}) = \frac{\partial \mathbf{z}}{\partial \pmb{\epsilon}} = \mathrm{diag}(\pmb{\sigma})
\end{equation}
With (XX), we can thus compute the log-posterior density by:
\begin{equation}
\begin{split}
\log q_{\pmb{\phi}}(\mathbf{z}|\mathbf{x}) & = \log p(\pmb{\epsilon}) - \log \left|\det \left(\frac{\partial}{\partial \pmb{\epsilon}}f(\pmb{\epsilon}, \pmb{\phi}, \mathbf{x})\right)\right| \\
& = \log p(\pmb{\epsilon}) - \sum_i \log \pmb{\sigma}_i \\
& = \sum_i \left(\log \mathcal{N}(0, 1)(\pmb{\epsilon}_i) - \log \pmb{\sigma}_i \right)
\end{split}
\end{equation}
As for the decoder, we can just insert the respective densities to compute $\log p_{\pmb{\theta}}(\mathbf{x}|\mathbf{z})$. For a normal distribution like above and $N$ the dimension of our input datapoints, we have:
\begin{equation}
\begin{split}
\log p_{\pmb{\theta}}(\mathbf{x}|\mathbf{z}) 
& = \log \left( (2\pi)^{-\frac{N}{2}}det(\pmb{\sigma}^2 E)^{-\frac{1}{2}} e^{-\frac{1}{2}(\mathbf{x}-\pmb{\mu})^T(\pmb{\sigma}^2 E)^{-1}(\mathbf{x}-\pmb{\mu})}\right) \\
& = - \frac{1}{2} \left( N \log2\pi + \sum_{i=1}^N \log\pmb{\sigma}_i^2 + \sum_{j=1}^N \frac{(\mathbf{x}_j - \pmb{\mu}_j)^2}{\pmb{\sigma}_j^2} \right) \\
& = - \frac{1}{2} \sum_{i=1}^N \left(\log(2\pi\pmb{\sigma}_i^2) + \frac{(\mathbf{x}_i - \pmb{\mu}_i)^2}{\pmb{\sigma}_i^2} \right) \\
\end{split}
\end{equation}
And for a Bernoulli distribution we can derive in the same way:
\begin{equation}
\begin{split}
\log p_{\pmb{\theta}}(\mathbf{x}|\mathbf{z}) 
& = \log \left(\prod_{i=1}^N \mathbf{p}_i^{\mathbf{x}_i}(1-\mathbf{p}_i)^{1-\mathbf{x}_i}\right) \\
& = \sum_{i=1}^N \mathbf{x}_i \log(\mathbf{p}_i) + (1 - \mathbf{x}_i) \log(1 - \mathbf{p}_i)
\end{split}
\end{equation}
And with $p(\mathbf{x}, \mathbf{z}) = p(\mathbf{x}|\mathbf{z})p(\mathbf{z})$, we have now all the tools to compute the estimator of the ELBO $\tilde{\mathcal{L}}_{\pmb{\theta}, \pmb{\phi}}(\mathbf{x}, \pmb{\epsilon})$ and its gradients and thus all the results to make AEVB work. With this setup of a Gaussian prior and variational encoder, as well as a Gaussian/Bernoulli generative model parametereized by ANNs, we call this instance of AEVB algorithm the \emph{Variational Autencoder} (VAE). Before we dive into the implementation and results of this thesis, we will discuss some further theoretical results that are more specific to the application of the VAE on the problem dataset. 


\section{Convolutional Neural Networks (CNN)}
While standard, Fully Connected Neuronal networks offer one way to realize the encoder and decoder of our VAE, there are approaches that work better on certain kinds of data. \emph{Convolutional Neural Networks} (CNNs) are an ANN structure, that offers good results when applied on high-dimensional image data. Just like ANNs, CNNs are organized in layers, that can be divided into \emph{Convolutions} and \emph{Pooling Layers}.
The main idea with CNNs is that images consist of recurring combinations of shapes that can be learned by lower dimensional \emph{filters} we apply to multiple positions in the image. This is why the input of CNN-layers is not just a flat vector of values like with ANNs, but instead a \emph{feature map} $\mathbf{A} \in Mat(d \times d \times l, \ \mathbb{R})$ with $d, l \in \mathbb{N}_{>0}$ that resembles the 2-dimensional structure of images in the first two dimensions. Feature maps that are not input or output of a CNN usually have $l>1$. This third dimension resembles multiple different filtered slices of the same image. We can thus think of $\mathbf{A}$ as a stack of 2D-images.

A convolution layer $\alpha^{(i)}$ is defined by the number of filters $n^{(i)} \in \mathbb{N}$, the \emph{filter size} $r^{(i)}\in \mathbb{N}$, \emph{stride} $s^{(i)} \in \mathbb{N}$ and activation $g^{(i)}: \mathbb{R} \rightarrow \mathbb{R}$. For each filter, we have a filter matrix $K^{(i)}_j \in Mat(r^{(i)} \times r^{(i)}, \mathbb{R})$, $j \in \{ 1, ..., n^{(i)}\}$,  that is multiplied element-wise to the input feature map at different locations of the input feature map. The results of these multiplications are summed and makes up for one value in the output map. Now, the locations to where the filter is applied in this way, can be best understood, if we think of the filter as a window, that slides over the input feature map in steps of size $s^{(i)}$. We start in the upper left corner in the image and begin our way horizontally to the left until $W^{(i)}_j$ overlaps the right edge of the input map and cannot be applied anymore. In this case, we move one step down and apply $W^{(i)}_j$ in the same way from left to right until we have applied it to the whole input map. To each of the outputs we apply $g^{(i)}$ and store the values in the same 2D activation structure.

\section{Double ELBO optimization}
When working with real world data such as images, we're usually confronted with a mix of layers of information that are in focus and others that are not. We want to refer to the former as foreground and the latter as background. For various applications it is desireable to seperate foreground and background signal in such data. Taking this as motivation, we want to expand our Variational Autoencoder in order to perform such a task.
So far, we have assumed that the generation process of our datapoints $\mathbf{x}$ involved only one multi-dimensional latent variable and one distribution. In our scenario, $\mathbf{x} = f(\mathbf{x}_1, \mathbf{x}_2)$ our data is a combination $f: \mathbb{R}^{N} \prod \mathbb{R}^N \rightarrow \mathbb{R}^N$ of foreground signal $\mathbf{x}_1$ and background signal $\mathbf{x}_1$.
We can consider two seperate latent variable models and have variables $\mathbf{z}_1$ and $\mathbf{z}_2$, that are involved in the generation of $\mathbf{x}_1$ and $\mathbf{x}_2$ respectively. Now, in our setup, we don't have direct access to $\mathbf{x}_1$ and $\mathbf{x}_2$ and we assume to have no means of inverting $f$ in any way. Maximizing the combined likelihood for $\mathbf{x}_1$ and $\mathbf{x}_2$ directly is thus not possible.

As for the background, we can approximate $\mathbf{x}_2$ by a scaled-down, averaged version of the data in order to take away the bias introduced by the foreground signal $\mathbf{x}_1$. We choose $g: \mathbb{R}^{MxM} \rightarrow \mathbb{R}^{mxm}$ with $M=ms$ for some $M, m, s \in \mathbb{N}$ to be a pooling function with stride $s$, that pools sectors of size $s$ to a single pixel creating a much smaller image where mostly large-scale background activity is retained. As we will see in the next chapter, this leads to quiet good results with our data. 
The foreground signal $\mathbf{x}_1$ is not as simple to approximate sufficiently. A pragmatic solution is to optimize the foreground likelihood for $\mathbf{x}$ instead of $\mathbf{x}_1$ and take the bias we get from the background signal as necessary evil.

For both latent spaces, we assume different variational encoders, $q_{\mathbf{\phi}}(\mathbf{z}_1|\mathbf{x})$ and $s_{\mathbf{\chi}}(\mathbf{z}_1|\mathbf{x})$, that are realized as CNNs. In order to account for the different nature of sharp and local foreground signal and blurry, large-spanning background signal, we choose different sizes for the receptive fields of the two encoders. For $q_{\mathbf{\phi}}(\mathbf{z}_1|\mathbf{x})$, our 'foreground encoder', we choose a small receptive field with a size similar to the regions of interest we want to capture in the foreground. Accordingly, we let $s_{\mathbf{\chi}}(\mathbf{z}_2|\mathbf{x})$ be parametrized by a CNN with large receptive field. With this setup, we basically have two VAEs that optimize two different ELBOs that we will derive in a bit. In order to allow for interaction between the two latent spaces, we expand the VAEs with one fully connected ANN layer connecting both spaces before feeding its output back into the seperate decoders. In other words, we alter our VAE model and have:
\begin{equation}
\begin{split}
& \mathbf{z}_1 \sim q_{\mathbf{\phi}}(\mathbf{z}_1|\mathbf{x}) \\
& \mathbf{z}_2 \sim s_{\mathbf{\chi}}(\mathbf{z}_2|\mathbf{x}) \\
& \tilde{\mathbf{z}}_1, \tilde{\mathbf{z}}_2 = NeuralNet((\mathbf{z}_1, \mathbf{z}_2)) \\
& \hat{\mathbf{x}} \sim p_{\mathbf{\theta}}(\mathbf{x}|\tilde{\mathbf{z}}_1) \\ 
& \hat{g(\mathbf{x})} \sim r_{\mathbf{\iota}}(g(\mathbf{x}|\tilde{\mathbf{z}}_2)
\end{split}
\end{equation}
From this, we can derive two ELBOs analog to (XX), that can be optimized in order to maximize $\log p_{\mathbf{\theta}}(\mathbf{x})$ and $\log r_{\mathbf{\iota}}(g(\mathbf{x}))$ for our data:
\begin{equation}
\begin{split}
 & \log p_{\mathbf{\theta}}(\mathbf{x}) + \log r_{\mathbf{\iota}}(g(\mathbf{x}))   \\
	= \ & \mathbb{E}_{q_{\mathbf{\phi}}(\mathbf{z}_1|\mathbf{x})}\left[\log p_{\mathbf{\theta}}(\mathbf{x})\right] + \mathbb{E}_{s_{\mathbf{\chi}}(\mathbf{z}_2|\mathbf{x})}\left[\log r_{\mathbf{\iota}}(g(\mathbf{x}))\right] \\
	= \ & \mathbb{E}_{q_{\mathbf{\phi}}(\mathbf{z}_1|\mathbf{x})}\left[ \log \frac{p_{\mathbf{\theta}}(\mathbf{x}, \tilde{\mathbf{z}}_1)}{p_{\mathbf{\theta}}(\tilde{\mathbf{z}}_1|\mathbf{x})} \right] 
	+ \mathbb{E}_{s_{\mathbf{\chi}}(\mathbf{z}_2|\mathbf{x})}\left[ \log \frac{r_{\mathbf{\iota}}(g(\mathbf{x}), \tilde{\mathbf{z}}_2)}{r_{\mathbf{\iota}}(\tilde{\mathbf{z}}_2|g(\mathbf{x}))} \right] \\
	= \ & \mathbb{E}_{q_{\mathbf{\phi}}(\mathbf{z}_1|\mathbf{x})}\left[ \log \left( \frac{p_{\mathbf{\theta}}(\mathbf{x}, \tilde{\mathbf{z}}_1)}{q_{\mathbf{\phi}}(\mathbf{z}_1|\mathbf{x})}\frac{q_{\mathbf{\phi}}(\mathbf{z}_1|\mathbf{x})}{p_{\mathbf{\theta}}(\tilde{\mathbf{z}}_1|\mathbf{x})}\right) \right] 
	+ \mathbb{E}_{s_{\mathbf{\chi}}(\mathbf{z}_2|\mathbf{x})}\left[ \log \left( \frac{r_{\mathbf{\iota}}(g(\mathbf{x}), \tilde{\mathbf{z}}_2)}{s_{\mathbf{\chi}}(\mathbf{z}_1|\mathbf{x})}\frac{s_{\mathbf{\chi}}(\mathbf{z}_1|\mathbf{x})}{r_{\mathbf{\iota}}(\tilde{\mathbf{z}}_2|g(\mathbf{x}))} \right) \right] \\
	= \ & \mathbb{E}_{q_{\mathbf{\phi}}(\mathbf{z}_1|\mathbf{x})}\left[ \log\left(\frac{p_{\mathbf{\theta}}(\mathbf{x}, \tilde{\mathbf{z}}_1)}{q_{\mathbf{\phi}}(\mathbf{z}_1|\mathbf{x})}\right) \right] + KL(q_{\mathbf{\phi}}(\mathbf{z}_1|\mathbf{x}) || p_{\mathbf{\theta}}(\tilde{\mathbf{z}}_2| \mathbf{x})) \\
	& + \mathbb{E}_{s_{\mathbf{\chi}}(\mathbf{z}_1|\mathbf{x})}\left[ \log\left(\frac{r_{\mathbf{\iota}}(g(\mathbf{x}), \tilde{\mathbf{z}}_2)}{s_{\mathbf{\chi}}(\mathbf{z}_2|\mathbf{x})}\right) \right] + KL(s_{\mathbf{\chi}}(\mathbf{z}_2|\mathbf{x}) || r_{\mathbf{\iota}}(\tilde{\mathbf{z}}_2| g(\mathbf{x})) \\
= \ & ELBO_1 + ELBO_2 + KL(q_{\mathbf{\phi}}(\mathbf{z}_1|\mathbf{x}) || p_{\mathbf{\theta}}(\tilde{\mathbf{z}}_2| \mathbf{x})) + KL(s_{\mathbf{\chi}}(\mathbf{z}_2|\mathbf{x}) || r_{\mathbf{\iota}}(\tilde{\mathbf{z}}_2| g(\mathbf{x}))
\end{split}
\end{equation}
Like before, with a single ELBO and VAE, we can obtain the same estimators of $ELBO_1$ and $ELBO_2$ and can thus apply AEVB algorithm in the same way. To simplify the notation for the practical discussion in the following, we want to refer to the function of calculating the output of the first VAE that optimizes $ELBO_1$ for an input $\mathbf{x}$ as $vae1(\mathbf{x})$ and in analog $vae2(\mathbf{x})$. Note, that these are not deterministic functions as they each include the sampling of a value for the latent variables. Also, because the latent spaces interact, each function includes the computation for both encoders.

\chapter{Implementation and Experiments}
For this thesis, three artificial datasets with abstract fore- and background signal were provided to evaluate the performance of the approaches described in the theoretical part. In a very abstract sense, the data tries to resemble real world videos of neural activity within mice brain obtained using a technique called \emph{2-photon excitation microscopy}, where a special camera is used to film multiple layers of neural cells. Each image has a very tight spatial focus on one layer of cells, making it vulnerable to background signal emmited by cells in layers that are not within focus. Their activity usually pans out across large sectors of the image making it harder to gain information about foreground signal by the cells located in the focussed layer. The real world data in this scenario suffers from many other problems that have to be addressed as well by preprocessing steps and more complex models. Also dimensionality is usually very high, requiring expensive training. 
Therefore, the artificial datasets try to emulate a similar, yet less challenging scenario. They each consist of 2000 images with 60x60 pixels and one channel of values. On each image, there are ten cells of a size of 6x6 pixels, that are independent and active (value equals 0.9 in every pixel) with a probability of 0.5 and else not active (value of zero). On top of that, we add noise that is normally distributed with mean $0.3$ and variance $0.4^2$ pixel-wise. To simulate background activity, we introduce a bigger, roundly shaped cell, that spans with a diameter of 30 pixels and where every pixel has a value of 0.6. This is where the three datasets vary: In the first dataset $\mathbf{X}_{90}$ we have one background cell in the bottom left corner that is active with a probability of 0.9 and overlaps three cells in the foreground. In the second dataset $\mathbf{X}_{50}$, the background cell is located at the same position but is active with only a probability of 0.5. In the third dataset $\mathbf{X}_{90, 20}$, we have a second background cell in the top right corner that overlaps two cells and is active with a probability of 0.2 in addition to the cell from dataset $\mathbf{X}_{90}$. As we want to focus on the evaluation of the VAEs on binary data with a Bernoulli decoder, which is why we binarize the data. As a result, most of the noise is retained in the background activity, while the foreground cells and the  in regions without activity are denoised nearly completely by this step. To get a better intuitive understanding of the data, consider figures (XX), (XX) and (XX). In the following, we want to compare the performance of dual VAEs, like described before, trained on the three datasets. We will first give some oversight over the implementation details and then present and discuss the performance of each of the three experiments.

\section{Implementation}
The algorithms have been implemented using the programming language Julia (version 1.2). Julia comes with a package called Flux, which provides most of the necessary methods to efficiently implement VAEs. 
As for the architecture, we used a dual VAE setup with $vae1$ for the foreground and $vae2$ for the background like described in section (XX). The encoders of both are realised as CNNs as well as the decoder of $vae1$. The decoder of $vae2$ will only have to fit a scaled target, which is why we chose a simple ANN decoder. For the scaling function for the target is:
\begin{equation}
g(\mathbf{x}) = sigmoid(...)
\end{equation}
The latent space of $vae1$ has four dimensions, while we chose it one dimensional for $vae2$. In order to match the targets in the data, we set the size of the receptive field for the foreground to 6 and for the background to 30. Because the ELBO estimators are combined by adding them up, we need to upscale the estimator of $ELBO_2$ as values here a much lower. We chose to multiply the correponding estimator by a factor of 450 although generally, values from 100 to 1000 seem to deliver comparable results.
Several training runs showed, that after around 15-20 epochs, the algorithm started to converge on all three datasets and the reconstruction plots did not show any improvement for both VAEs. To reduce variability of the results between the three datasets, we therefore opted to train the algorithm for 30 epochs on each dataset.
Further details regarding implementation can be retrieved from the code that is provided with this work.

The plots we will discuss in the following have been produced with the Plots library. We have two types of plots:
\begin{itemize}
\item[1.] Reconstruction plots, where we plot the outputs of both VAEs $vae1(\mathbf{x})$, $vae2(\mathbf{x})$ for an input image $\mathbf{x}$ together with its scaled version $g(\mathbf{x})$.
\item[2.] Latent space exploration plots, that plot the outputs of the decoder of $vae1$ for a number of different configurations of the latent variables. Configurations are vectors with one 'hot' dimension of a value of $\pm a \in \mathbb{Z}$ and remaining dimensions set to zero.
\end{itemize}
Both types of plots will provide us with the necessary tools to evaluate performance and compare it visually between the different datasets.

\section{Results}
We will first have a look at the reconstruction plots for all three scenarios to gain a first insight on the performance of the three VAEs.
If we compare figures \ref{fig:recon1}, \ref{fig:recon2} and \ref{fig:recon3}, we can see that all three algorithms reconstruct the scaled target for $vae2$ without any visible errors. Also, the reconstruction provided by $vae1$ matches shape and activity of the background across all three experiments, although the values of the pixels are lower $1$. At the same time, the reconstruction of the foreground cells seems to differ in quality. 

In case of training on dataset $\mathbf{X}_{90}$, figure \ref{fig:recon1} shows that the reconstruction of the foreground in $vae1$ seems to work fairly well on the sample. The background cell is active in seven out of the ten pictures with varying noise. Most of the foreground activity has been matched and we have misses only around the area of the background cell. Here, the algorithm seems to have ploblems to reconstruct correctly and to some extent embeds the foreground activity into the background, which can be observed as slightly lighter spots within the cell. Furthermore, there is one cell that is located on the right edge of the background cell, which seems to to be active in every reconstructed image. The appearance of these spots seems unaffected by the actual activity of the foreground cells they represent. However, the last row of the plots, which shows an image with inactive background, gives evidence, that at least some level of separation takes place. We can see that besides one foreground cell, which appears to be a false positive, all other active foreground cells also show activity in the reconstructed image.

With the intoduction of a second background cell in dataset $\mathbf{X}_{90, 20}$, the situation looks similar in the reconstruction plots (figure \ref{fig:recon3}). Again, active foreground cells without contact to background activity are reconstructed without error. This situation changes if we look at cells that overlap with background cells. In the third image, we can see, that the top right foreground cell shows only very minor activity in the reconstructed image. However, other cells that overlap with the top left background cell seem to be reconstructed well in the case of activity and inactivity. This is in contrast to the cells located around the bottom left background cell. Like with the dual VAE trained on $X_{90}$, the reconstruction of these cells shows mixed results. Also, we can again observe, that the activity of foreground cells is not reconstructed perfectly in case of inactivity of the background, although all of the active cells show at least some activity. Adding to this, the foreground cells also seem to be embedded in the background cell regardless of their activity in the input. Interestingly, we can only observe this effect in the lower left background cell.

If we decrase the probability of the background cell like in dataset $\mathbf{X}_{50}$, figure \ref{fig:recon2} shows, that the reconstruction of $vae1$ has more false positives outside of the area of the background cell. This is due to the fact, that in all reconstructed images, the bottom left and right foreground cells are constantly active with close to constant level throughout all images. Interestingly, the cells that overlap with the background cell show much better results than both of the other scenarios. The remaining foreground cells are reconstructed correctly across the whole sample.

Next, we want to consider figures \ref{fig:expl1}, \ref{fig:expl3} and \ref{fig:expl2}. We chose a value of $a = \pm 5$ for these plots by observing the latent space for multiple input images, where the maximum value for a each single dimension was around that value.

Beginning with figure \ref{fig:expl1}, it is observable, that all dimensions of the latent space seem to contain some information about the background activity. For example a state of $\mathbf{z}_1 = (5, 0, 0, 0)$ or $\mathbf{z}_1 = (0, 5, 0, 0)$ and $\mathbf{z}_2 = 0$ decodes to a very strong background cell. On the other hand, not all parts of the space contain background activity. Especially for $\mathbf{z}_1 \in \{(-5, 0, 0, 0), (0, -5, 0, 0), (0, 0, 0, 5)\}$ and $\mathbf{z}_0 = 0$, we even can find the cells that are overlapped by the background without much distortion. 

A similar result can be found for the scenario of two background cells in figure \ref{fig:expl3}, where even more parts of the latent space contain information about the background cells. Here we also have parts, that are decoded to images, where the foreground cells, that are overlapped by background activity, appear more or less sharp. For example $\mathbf{z}_1 = (0, -5, 0, 0)$ and $\mathbf{z}_2=0$ or $\mathbf{z}_1 = (0, 0, 0, 0)$ and $\mathbf{z}_2 = -5$. Some of the plots also show areas with values close to zero at the locations of foreground cells within the background cells that are decoded.

In contrast to the other two scenarios, in figure \ref{fig:expl2} things look a bit differently. Most of the dimensions contain exclusively information about foreground cells with only minor distortion at cells that are overlapped by the background cell. Only for states $\mathbf{z}_1 = (0, 0, -5, 0)$ and $\mathbf{z}_2=0$ or $\mathbf{z}_1 = (0, 0, 0, 0)$ and $\mathbf{z}_2 = 5$, we get decoded images with background activity. Also, the bottom left and right foreground cells, that we observed before, are active in every part of the latent space we plottet in the exploration. This also differs from the other scenarios, where we had at most one component, that is active in every part of the space. ALWAYS ACTIVE PARTS DESCRIBE


 

\begin{figure}
\begin{center}
  \includegraphics[width=11.5cm]{../code/plots/reconstruction1.png} 		  \includegraphics[width=1.5cm]{../code/plots/scale.png}
  \caption{Reconstruction plot of eight images $\mathbf{x}$ sampled from dataset $\mathbf{X}_{90}$. Each row resembles a different input image and we have columns for the input image $\mathbf{x}$, output $vae1(\mathbf{x})$, the scaled input image $g(\mathbf{x})$ and output $vae2(\mathbf{x})$.}
  \label{fig:recon1}
\end{center}
\end{figure}

\begin{figure}
\begin{center}
  \includegraphics[width=11.5cm]{../code/plots/reconstruction3_48_55.png}
  \includegraphics[width=1.5cm]{../code/plots/scale.png}
  \caption{Reconstruction plot of eight images $\mathbf{x}$ sampled from dataset $\mathbf{X}_{90, 20}$. Each row resembles a different input image and we have columns for the input image $\mathbf{x}$, output $vae1(\mathbf{x})$, the scaled input image $g(\mathbf{x})$ and output $vae2(\mathbf{x})$.}
  \label{fig:recon3}
\end{center}
\end{figure}

\begin{figure}
\begin{center}
  \includegraphics[width=11.5cm]{../code/plots/reconstruction2_1_8.png}
  \includegraphics[width=1.5cm]{../code/plots/scale.png}
  \caption{Reconstruction plot of eight images $\mathbf{x}$ sampled from dataset $\mathbf{X}_{50}$. Each row resembles a different input image and we have columns for the input image $\mathbf{x}$, output $vae1(\mathbf{x})$, the scaled input image $g(\mathbf{x})$ and output $vae2(\mathbf{x})$.}
  \label{fig:recon2}
\end{center}
\end{figure}

\begin{figure}
\begin{center}
  \includegraphics[width=7cm]{../code/plots/exploration1.png}
  \includegraphics[width=1.5cm]{../code/plots/scale.png}
  \caption{Exploration of latent space of a dual VAE trained on dataset $\mathbf{X}_{90}$. Images show outputs of $vae1$ for different states of $\mathbf{z}_1$ and $\mathbf{z}_2$, which are annotated below the respective output in the form [$\mathbf{z}_{1, 1}$, $\mathbf{z}_{1, 2}$, $\mathbf{z}_{1, 3}$, $\mathbf{z}_{1, 4}$, $\mathbf{z}_2$].}
  \label{fig:expl1}
\end{center}
\end{figure}

\begin{figure}
\begin{center}
  \includegraphics[width=7cm]{../code/plots/exploration3.png}
  \includegraphics[width=1.5cm]{../code/plots/scale.png}
  \caption{Exploration of latent space of a dual VAE trained on dataset $\mathbf{X}_{90, 20}$. Images show outputs of $vae1$ for different states of $\mathbf{z}_1$ and $\mathbf{z}_2$, which are annotated below the respective output in the form [$\mathbf{z}_{1, 1}$, $\mathbf{z}_{1, 2}$, $\mathbf{z}_{1, 3}$, $\mathbf{z}_{1, 4}$, $\mathbf{z}_2$].}
  \label{fig:expl3}
\end{center}
\end{figure}

\begin{figure}
\begin{center}
  \includegraphics[width=7cm]{../code/plots/exploration2.png}
  \includegraphics[width=1.5cm]{../code/plots/scale.png}
  \caption{Exploration of latent space of a dual VAE trained on dataset $\mathbf{X}_{50}$. Images show outputs of $vae1$ for different states of $\mathbf{z}_1$ and $\mathbf{z}_2$, which are annotated below the respective output in the form [$\mathbf{z}_{1, 1}$, $\mathbf{z}_{1, 2}$, $\mathbf{z}_{1, 3}$, $\mathbf{z}_{1, 4}$, $\mathbf{z}_2$].}
  \label{fig:expl2}
\end{center}
\end{figure}


\section{Discussion}
We saw, that there is a lot of differences between the results of the three scenarios we evaluated above. 
The levels of the background cells were lowered in the outputs of $vae1$ across all datasets. This is a result of the noise, which forces the algorithm, that tries to maximize $log p_{\pmb{\theta}}(\mathbf{x})$, select a lower optimal value. However we saw, that some regions within the background cells with a 90\%-activity rate had higher levels. This makes sense, as these were regions, where a foreground cell overlapped and thus neutralized the noise in case of activity embedding some of their activity into the background. Also, some of the overlapping foreground cells were decoded as active in all dimensions of the exploration plots. In this case, we argue that these cells were separated to some extend from the background cell instead of just being completely embedded into the background signal like others.

Additionaly, foreground cells that were located within 90\% active background cells, were usually reconstructed much worse. Especially, if the background cell is inactive, we could see, that the reconstructed cells were distorted and showed only light activity. As some of the cells were just decoded as constantly active, we also have a high false positive rate in this case. When the background is active, we can not identify, whether the foreground cell is reconstructed as active or if we just see the background cell with embedded portions of the foreground cell.
Both problems are most probably a symptom to the fact that the background signal obstructs the foreground signal in around 9 out of 10 pictures. Hence, the data is heavily weighted towards images, where the foreground cells, that are located within the range of the activity, are poorly visible to the algorithm. As we optimize over the loss of all images, it is cheaper for the algorithm to not reconstruct these cells well and be slurry on their exact activity and by that gaining headroom to reconstruct other components more correctly. Some of this activity was for example regarded as part of the background cell, like we just argued, leaving more parts of the latent space for information about other activity. Several solutions might be interesting to tackle this problem. First, we could try to upsample images with low background activity in the dataset. This would make it more expensive for the algorithm to reconstruct the cells that are obstructed incorrectly. Also, we could allow more latent dimensions adding more complexity to the model and possibly giving the algorithm the means to capture the details of obstructed foreground cells better.

Contrary to the intuition we used set up the double VAE structure and choose a big receptive field in $vae2$, the information about background cells is spread across multiple dimensions in the latent spaces of models trained on datasets with high background activity of 90\%. Reasons for this might be the choice of a less complex encoder in $vae2$ or the low dimensionality of 1 of its latent space. 
In the scenario of a less active background cell, this problem wasn't as observable. Although we saw in figure \ref{fig:recon2}, that the information on the background cell could be found in one dimension of the latent space of $vae1$ in addition to the latent space of $vae2$, the remaining parts of the space were free of background information. In this case, the double VAE setup did an even better job at separating fore and backround signal on the latent level. However, we also saw, that the algorithm seemed to perform worse on cells, that were not located in the areas of background signal in comparison to the other tow scenarios. At second sight, one can realize, that the number of correctly reconstructed foreground cells might in fact be higher or at least equal. In the scenarios with highly active background, the algorithm usually did not retain most information about the foreground cells within or on the edges of the background cells. As a result, the algorithm had more headroom on the level of the latent space to reconstruct cells outside of these regions better. However in the case of a less active background, we can see in both the scenario with 50\% backround activity and the scenario with a second 20\% active background, that the foreground cells overlapping these background cells were reconstructed much better. As the background in these cases is inactive much more often, it gets more expensive for the algorithm to ignore the activity. This probably leads to more parts of the latent space being used to retain information about these foreground cells taking away parts that could have been used for the cells outside of background activity. We suspect, that this might be the reason for the slightly worse performance on foreground cells outside of background activity in said scenario. Again, it would be promising to do further research and comparing our reconstruction results to approaches with higher dimensionality on the latent level. 




 
\medskip

\begin{thebibliography}{9}

\bibitem{kingma1} 
D. Kingma, M. Welling: 
\textit{Auto-Encoding Variational Bayes}. 
arXiv: 1312.6114v10 [stat.ML] (2014).

\bibitem{kingma2} 
D. Kingma, M. Welling: 
\textit{An Introduction to Variational Autoencoders}. 
arXiv: 1906.02691 [cs.LG] (2019).

\bibitem{cvae} 
...: 
\textit{Auto-Encoding Variational Bayes}. 
arXiv: 1312.6114v10 [stat.ML] (2014).

\bibitem{KL} 
M. de Queiroz, R. Silva, R. Loschi: 
\textit{Shannon Entropy and Kullback-LeiblerDivergence in Multivariate Log FundamentalSkew-Normal and Related Distributions}. 
arXiv: 1408.4755 [math.ST] (2016).

\bibitem{arithmetic} 
Vincent Dumoulin, Francesco Visin: 
\textit{A guide to convolution arithmetic for deep learning}. 
arXiv:1603.07285 [stat.ML] (2016).

\bibitem{stanford} 
Stefano Ermon: CS 228 - Probabilistic Graphical Models.
\\\texttt{https://ermongroup.github.io/cs228-notes/}
\end{thebibliography}

\end{document}